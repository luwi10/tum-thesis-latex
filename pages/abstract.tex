\chapter{\abstractname}

Imaging algorithms have important applications in our daily life. 
They are used in medical imaging, security checks, and non-destructive testing. 
A very common algorithm for microwave imaging is the backprojection algorithm, which is the backbone of modern airport scanners.
This algorithm only considers the distances between the receiver and the imaging point to construct the image from the measured signals.
Time-reversal imaging is a more advanced algorithm.
It excels in the application in multipath environments, where the signal is reflected by multiple surroundings.
Unlike the backprojection algorithm, time-reversal also takes the surroundings into account.
This results in a significantly better image quality, especially when no line-of-sight between the receivers and the imaging domain exists.
On the downside, time-reversal is often based on full-wave simulations that are computationally expensive.
In this thesis, a new approach to improve the time-reversal algorithm for microwave imaging is presented.
To do this the ray concept of the electromagnetic field that arises from geometrical optics is used.
This way the ray-tracing capabilities that are implemented in modern GPUs can be used to speed up the time-reversal algorithm, especially for multipath environments.
The algorithm was implemented using the ray-tracing capabilities of the NVIDIA OptiX framework.
For this approach, all possible paths of the waves are computed and stored as wavefronts in a first run.
In a second run, the time-reversal image for every wavefront is computed, which can then be combined to create the final image.
For several scenarios, the algorithm was tested and compared to the traditional naive backprojection algorithm.
The results show that it can significantly improve the image quality compared to the traditional approach.

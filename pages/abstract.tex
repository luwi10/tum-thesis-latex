\chapter{\abstractname}

Imaging algorithms have important applications in our daily life. They are used in medical imaging, security checks, and non-destructive testing. 
A very common algorithm for microwave imaging is the time-reversal algorithm.
This algorithm is often based on full-wave simulations that are computationally expensive.
In this thesis, a new approach to improve time-reversal algorithms for microwave imaging is presented.
To do this the ray concept of the electromagnetic field that arises from geometrical optics is used.
This way the ray-tracing capabilities that are implemented in modern day GPUs can be used to speed up the time-reversal algorithm, especially for complex multipath environments.
The algorithm was implemented using the ray-tracing capabilities of the NVIDIA OptiX framework.
The results show that it can improve the image quality and reduce the computational time compared to the conventional time-reversal algorithm.
For the future work, some ideas are proposed to further improve the algorithm and make it more robust for different scenarios.
For instance the algorithm could be extended to also take the polarization of the electromagnetic field in consideration.
With these improvements the algorithm could be used in real-world applications for example for the localization of wifi routers in a building.

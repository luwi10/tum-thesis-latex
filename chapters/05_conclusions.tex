% !TeX root = ../main.tex
% Add the above to each chapter to make compiling the PDF easier in some editors.

\chapter{Conclusions and Outlook}\label{chapter:conclusions}
The thesis presented a ray tracing approach for calculating the time-reversal image of a specified domain from measured data in the microwave spectrum.
The needed theoretical background was summarized and the algorithm was tested on a set of different examples.
The results showed, that the algorithm is able to improve the image quality by considering the multipath propagation from a surrounding setup in some scenarios, especially when the direct path is blocked.
As the greatest disadvantage of the algorithm, the high computational cost was identified.
Although using a GPU for the calculations, no real-time application is possible for practical cases.
But the algorithm is still much faster than a full-wave simulation and can be used for a first approximation of the time-reversal image.
A possible field of further research could be the optimization of the algorithm for real-time applications, by improving the placement of the receivers and the reflector plates, so that the number of needed rays can be reduced, while still maintaining a good image quality.
Another possible improvement could be the use of a more sophisticated ray tracing algorithm, that can handle more complex geometries and materials, as in this thesis only plane PEC-plates were used.
Such an algorithm would need to treat the electromagnetic wave as a vector field instead of a simple scalar field, as the polarization of the wave influences the scattering and reflecting process.
Furthermore the performance of the algorithm with real-world measurements could be tested, as the results in this thesis were only based on simulated data.
The algorithm could also be extended to work with other types of waves, like acoustic waves or seismic waves, as the principle of time-reversal imaging is not limited to the microwave spectrum.
But for creating a real benefit for practical applications, a more advanced algorithm is needed, that can not only image active-sources, but also passive scatter-points.
For this purpose, the algorithm would also have to consider the locations of illuminating transmitters.




% !TeX root = ../main.tex
% Add the above to each chapter to make compiling the PDF easier in some editors.

\chapter{Background}\label{chapter:background}
In this chapter, the theoretical background of time-reversal imaging and rays as model for electromagnetic waves according to Maxwell's equations is presented.
These two concepts are essential for understanding the the time-reversal imaging process proposed in this thesis and its underlying physical principles. 

\section{Time-Reversal of the Electromagnetic Field}
\subsection{Time-Reversal in physics}
The evolution of a physical system in time can be described by a curve \(s(t)\) that is moving through a statespace \(\mathbb{P}\).
When talking about time-reversal in general, one is usually referring to the time-reversal transformation which consists of two components as stated in~\parencite{roberts_reversing_2022}.
First, the curve \(s(t)\) has to be time-reversed by \(s(t) \mapsto s(-t)\).
Secondly, a time-reversal operator has to be applied to each instantaneous state \(s(t) \mapsto \operatorname{T}s(t)\), i.e.\ a ball flying to the right would become a Ball flying to the left.
This time-reversal operator is reversing for some properties (momentum) and for some properties preserving (kinetic energy).


In summary, time reversing the curve \(s(t)\) results in the curve \(r(t)=\operatorname{T}s(-t)\). 



\subsection{Time-Reversal invariance of physical laws}
As mentioned before, the curve \(s(t)\) through the statespace \(\mathbb{P}\) describes the evolution of a physical system in time.
This means, \(s(t)\) is a solution to the equations given by the laws that describe the system. 
A law being time-reversal invariant, therefore implies that the time-reversed curve \(r(t)\) is also a solution to this law~\parencite{roberts_time_2021}.

For a more hands-on example, imagine filming a ball that is influenced by gravity flying through a vacuum.
Newtons second law \({F}=m \cdot {a}\) is time-reversal invariant. This means the film can be played backwards and the trajectory of the ball will still follow Newtons second law.

An example for a non time-reversal invariant law is the second law of thermodynamics, which states that the entropy of an isolated system never decreases.
This means, that a film of a gas spreading out in a room played backwards would not comply to this law, as the entropy of the system in the backwards film actually decreases by time.



\subsection{Time-Reversal invariance of Maxwell's equations}
Maxwell's equations formulate the foundations of classical electromagnetism.
They relate the electric field \(E\), the electric displacement field \(D\), the magnetic field \(H\) and the magnetic flux density \(B\)  by the following equations, where \(\rho \) and \({J}\) denote the electric charge and current densities.
\begin{align}
    \nabla \cdot D &= \rho \label{eq:maxwell1} \\
    \nabla \cdot B &= 0 \label{eq:maxwell2} \\
    \nabla \times E &= -\frac{\partial B}{\partial t} \label{eq:maxwell3} \\
    \nabla \times H &= J + \frac{\partial D}{\partial t} \label{eq:maxwell4}
\end{align}
Generally \(E\) and \(D\) are related by the electric permittivity \(\epsilon \), while \(H\) and \(B\) are related by the magnetic permeability \(\mu \).
In this paper only reciprocal anisotropic media are considered, which means that the permittivity \(\epsilon \) and the permeability \(\mu \) are hermitian tensors.~\parencite{krowne_electromagnetic_1984}
This leads to a reduced form of Maxwell's equations, where the electric field \(E\) and the magnetic field \(H\) are related by
\begin{align}
    \nabla \cdot E &= \frac{\rho}{\epsilon} \label{eq:maxwell5} \\
    \nabla \cdot H &= 0 \label{eq:maxwell6} \\
    \nabla \times E &= -\mu \frac{\partial H}{\partial t} \label{eq:maxwell7} \\
    \nabla \times H &= \epsilon {J} + \frac{\partial E}{\partial t} \label{eq:maxwell8}
\end{align}

For the time-reversal of these equations, one first has to understand, how the time-reversal operator \(\operatorname{T}\) acts on the fields \(E\) and \(H\).

Taking into account, that the Poynting vector \(S = E \times H\) is a measure for the energy flow of the electromagnetic field, the time-reversal of these fields has to be chosen in a way that the Poynting vector is reversed as well.
This means, that either \(E\) or \(H\), but not both, have to be reversed so the Poynting vector changes sign. 

According to \parencite{sigwarth_time_2022} the Maxwell equations remain invariant if the fields and densities are transformed as follows:
\begin{align}
    \operatorname{T}E &= E \\
    \operatorname{T}H &= -H \\
    \operatorname{T}\rho &= \rho \\
    \operatorname{T}{J} &= -{J}
\end{align}

An intiutive explanation for this transformation is given by the fact, that the electric field \(E\) is a measure for the force acting on a charge, while the magnetic field \(H\) is a measure for the force acting on a moving charge. 
Accordingly if we reverse time, the direction of the charges will revert and hence the magnetic field should revert. 
In contrast there is no reason for a positive charge to become a negative charge and vice versa, so the electric field stays the same.



\subsection{Time-Reversal in the frequency domain}
With the time-reversal operator defined for all components of Maxwell's equations we can now write the time-reversal of the electric and magnetic field in the time domain as
\begin{align}
    E_{TR}(t) &= E(-t) \\
    H_{TR}(t) &= -H(-t)
\end{align}
The fourier transformation of these equations with \(\mathcal{F}\{X(t)\} = \hat{X}(\omega) = \int X(t) \cdot e^{i\omega t} \, dt\) leads to
\begin{align}
    \hat{E}_{TR}(\omega) &= \hat{E}{(\omega)}^* \\
    \hat{H}_{TR}(\omega) &= -\hat{H}{(\omega)}^*
\end{align}
where \(^*\) denotes the complex conjugate. 

As only fields in the frequency domain will be considered in the following, the hat is dropped.



\section{Mathematical justification for the Time-Reversal of Electromagnetic waves}\label{sec:math_foundations_em_tr}
In \parencite{de_rosny_theory_2010} the mathematical groundwork for the time-reversal of electromagnetic waves has been laid out.
This chapter won't cover the derivation of the results, but review them as they represent the starting point of mathematical justification for the time-reversal of electromagnetic waves.

\subsection{The six-vector formalism}
The mentioned paper uses two formalisms.
The electric and magnetic field are combined into the  six-vector \(\psi \) which is defined as
\begin{equation}
    \psi = \begin{pmatrix}
        \mathbf{E} \\
        i\mathbf{H}
    \end{pmatrix}
\end{equation}
and a source vector \(q\) is defined as
\begin{equation}
    q = \begin{pmatrix}
        -i\mathbf{J} \\
        \mathbf{m}
    \end{pmatrix}
\end{equation}
with \(\mathbf{J}\) and \(\mathbf{m}\) denoting the electric and magnetic current density vectors.

These vector-distributions are linearly related through the Green's function \(G\) of the system
\begin{equation}
    \psi(\mathbf{r}) = \int G(\mathbf{r}, \mathbf{r}^{\prime}) q(\mathbf{r}^{\prime}) \, d^3 r^{\prime}
\end{equation}
For improved readability the integral is written in the following way
\begin{equation}
    \int G(\mathbf{r}, \mathbf{r}^{\prime}) q(\mathbf{r}^{\prime}) \, d^3 r^{\prime} = [G(\mathbf{r}) q]
\end{equation}

The above defintions imply that the time-reversal operator \(\operatorname{T}\) acts on the six-vector and the source vector by complex conjugation.
\begin{align}
    \operatorname{T}\psi &= \psi^* \\
    \operatorname{T}q &= q^*
\end{align}

\subsection{The electromagnetic time-reversal equation}
Under the assumption of a reciprocal environment (\(\epsilon \) and \(\mu \) are hermitian) the following equation is derived in \parencite{de_rosny_theory_2010}:
\begin{equation}\label{time-reversal-equation}
    \begin{aligned}
    & \underbrace{\oint_{\delta \Gamma} G(\mathbf{r}^{\prime}, \mathbf{r})\left(\mathbf{n}\left(\mathbf{r}^{\prime}\right) \times\left[G^*\left(\mathbf{r}^{\prime}\right) q^*\right]\right) d^2 r^{\prime}}_{\psi_S(\mathbf{r})} \\
    &=\underbrace{[G^*(\mathbf{r}) q^*] - [G(\mathbf{r}) q^*]}_{\psi_V(\mathbf{r})}
    \end{aligned}
\end{equation}
Taking into account that \([G^*(\mathbf{r}) q^*] = \psi^*(\mathbf{r}) = \operatorname{T}\psi(\mathbf{r})\), the left-hand side \(\psi_S(\mathbf{r})\) can be interpreted as a procedure of four steps:
\begin{enumerate}
    \item Measure the electromagnetic field \(\psi \) at the surface \(\delta \Gamma \).\ (\(\oint_{\delta \Gamma} \ldots  (\ldots \times \psi^*(\mathbf{r}^{\prime})) d^2 r^{\prime}\))
    \item Time-revere these measurements.\ (\(\psi^*(\mathbf{r}^{\prime}) = \operatorname{T}\psi(\mathbf{r}^{\prime})\))
    \item Deduce a source distribution \(q_{TR}\) from these time-reversed measurements on the surface \(\delta \Gamma \) according to the normal vector of the surface.\ (\(q_{TR}(\mathbf{r}^{\prime}) = \mathbf{n}(\mathbf{r}^{\prime}) \times \operatorname{T}\psi(\mathbf{r}^{\prime})\))
    \item Calculate the electromagnetic field \(\psi_S(\mathbf{r})\) generated by this source distribution.\ (\(\psi_S(\mathbf{r}) = \oint_{\delta \Gamma} G(\mathbf{r}^{\prime}, \mathbf{r}) q_{TR}(\mathbf{r}^{\prime})\))
\end{enumerate}
The electromagnetic time-reversal equation~\eqref{time-reversal-equation} then states that this field \(\psi_S(\mathbf{r})\) is equal to the time-reversed field \(\psi_{TR}(\mathbf{r}) = [G^*(\mathbf{r}) q^*]\) minus a diverging field \(A(\mathbf{r}) = [G(\mathbf{r}) q^*]\).
\begin{equation}
    \psi_S(\mathbf{r}) = \psi_{TR}(\mathbf{r}) - A(\mathbf{r})
\end{equation}
Further investigation of \(A\) shows that it is equal to the field generated by the time-reversed sources.
As the sources are initially turned off, their time-reversed excitation and thus the field \(A\) eventually goes to zero.

In conclusion \parencite{de_rosny_theory_2010} proves that it is possible to compute or recreate the time-reversed version of an electromagnetic field \(\psi_{TR}\) inside of a mediumg \(\Gamma \) by measuring the field \(\psi \) at the surface \(\delta \Gamma \), applying the time-reversal operator \(\operatorname{T}\) to these measurements and emiting the resulting sources back into \(\Gamma \).



\section{A signal theory approach to Time-Reversal Imaging}

\subsection{Active-Source location}
Assuming that the \(N\) point-sources, denoted as \(q_i(\omega)\), are located at positions \(\mathbf{r_i}\), we obtain the source-distribution \(q(\mathbf{r}, \omega)\).
\begin{equation}
    q(\mathbf{r}, \omega) = \sum_{i=1}^{N} q_i(\omega) \cdot \delta(\mathbf{r} - \mathbf{r_i})
\end{equation}
To explain how the described time-reversal process can be used for active-source location, a signal theory approach is taken.
To do this the medium is modelled as a reciprocal LTI-System.
Now the propagation of an electromagnetic wave-signal \(\psi(\mathbf{r}, \omega )\) from a source \(q_a(\omega )\) at \(\mathbf{r_a}\) to a receiver at \(\mathbf{r_b}\) can be reduced to a simple input-output system with the transfer function \(H_{a\rightarrow b}(\omega) = G(\mathbf{r_a}, \mathbf{r_b}, \omega)\).
\begin{equation}
    \psi(\mathbf{r_b}, \omega) = H_{a\rightarrow b}(\omega) \cdot q_a(\omega)
\end{equation}
This can be applied to the whole source distribution \(q(\mathbf{r}, \omega)\) to obtain the wave-signal at the receiver.
\begin{equation}
    \psi(\mathbf{r_b}, \omega) = \sum_{i=1}^{N} H_{i\rightarrow b}(\omega) \cdot q_i(\omega)
\end{equation}
From this wave-signal, an equivalent source \(p_b(\omega)\) can be deduced by applying the measurement operator \(\operatorname{M}\). This is done at all \(M\) receivers at the locations \(\mathbf{r_j}\).  
\begin{equation}
    p_j(\omega) = \operatorname{M} \psi(\mathbf{r_j}, \omega)
\end{equation}
Time-reversing (complex-conjugating) the measured sources at all the receivers and reemitting them will result in the following wave-signal distribution \(\psi_{TR}(\mathbf{r_x}, \omega)\) that is again measured with \(\operatorname{M}\)
\begin{equation}\label{signal-time-reversal}
    \operatorname{M} \psi_{TR}(\mathbf{r_x}, \omega) = \sum_{j=1}^{M} \operatorname{M} H_{j\rightarrow x} \cdot p_j^*(\omega) = \sum_{j=1}^{M} \sum_{i=1}^{N} \operatorname{M} H_{j\rightarrow x} \cdot \operatorname{M}^* H^*_{i\rightarrow j}(\omega) \cdot q^*_i(\omega)
\end{equation}
At this point it should be mentioned, that the usage of \(\psi \), \(q\) and the relating measurement operator \(\operatorname{M}\) was chosen for consistency with the Section~\ref{sec:math_foundations_em_tr}.
In most papers about electromagnetic time-reversal only the electric field \(E\) is considered for simplicity.
Then, by substituting \(E\) for \(\psi \) and describing the sources \(q_i\) by the electric field \(E_i\) they cause at location \(\mathbf{r_i}\), equation~\eqref{signal-time-reversal} becomes
\begin{equation}
    E_{TR}(\mathbf{r_x}, \omega) = \sum_{j=1}^{M} \sum_{i=1}^{N} H_{j\rightarrow x} \cdot H^*_{i\rightarrow j}(\omega) \cdot E^*_i(\omega)
\end{equation}

This approach is valid for the far-field application.
Here the electric and magnetic fields are orthogonal, so it is sufficient to only keep track of the electric field.
Furthermore the deduction of an equivalent source \(q_{TR}(\mathbf{r}) = n(\mathbf{r}) \times \psi^*(\mathbf{r})\) can be simplified to being proportional to the electric and magnetic field respectively~\parencite{de_rosny_theory_2010}
\begin{equation}
    q_{T R}(\mathbf{r})=\left(\begin{array}{c}
        \frac{i \mathbf{E}^*(\mathbf{r})}{\eta} \\
        \eta \mathbf{H}^*(\mathbf{r})
        \end{array}\right)
\end{equation}





\section{The Ray Concept in Electromagnetic waves}



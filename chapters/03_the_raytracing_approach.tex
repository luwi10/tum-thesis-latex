% !TeX root = ../main.tex
% Add the above to each chapter to make compiling the PDF easier in some editors.

\chapter{The ray tracing approach}\label{chapter:the_raytracing_approach}
In Chapter~\ref{chapter:background} the theoretical background for time-reversal imaging and the utilization of ray tracing for the needed wave-simulation were laid out.
This chapter will describe the implementation of a ray tracing approach for the time-reversal imaging algorithm.

As mentioned in \parencite{dyab_critical_2013} the quality of the time-reversed image increases within a highly reverberant environment.
In contrast to other imaging algorithms, multipath propagation of the waves is therefore not suppressed but used to improve the time-reversal imaging.
To simulate the multipath propagation, one is mostly interested in all the paths that the signal takes from the source to every point on the simulation domain.
This is why ray tracing is such a suitable approach for this problem.

\section{The imaging algorithm}
The time-reversal imaging algorithm is a three-step process.
First, the scene is illuminated with a monochromatic signal of frequency \(\omega_i\) coming from a source at location \(\mathbf{t}_x\) and the resulting scattered field is recorded at \(M\) receiver locations \(\mathbf{r}_j\). 
This step is repeated for \(N\) frequencies \(\omega_i\) resulting in the \(N \times M\) matrix \(\mathsf{S} \) containing the scattered field at all receiver locations for all frequencies.

In the second step, a random location in the target domain is chosen as the source of a big number of rays which are launched into all directions.
A ray that follows a possible path from the target location to a receiver location will eventually hit the receiver so the path can be recorded.
From these paths, the paths for the other locations in the target domain can be derived.
This step is further described in Section~\ref{section:virtual_targets}.

Finally, the measured scattered field \(\mathsf{S}\) is complex conjugated and sent back from each corresponding receiver to each point in the target domain along the recorded paths.
The sum of all these signals at each point in the target domain then gives rise to the time-reversed image of the target.
For a better resolution of the image, this step is repeated for all measured frequencies \(\omega_i\).
This can be summarized in the final imaging formula
\begin{equation}
    I(\mathbf{r}_k) = \sum_{i=1}^{N} \sum_{j=1}^{M} \mathsf{S}_{ik} \cdot H_{j \rightarrow k}(\omega_i)
\end{equation}
where \(H_{j \rightarrow k}(\omega_i )\) is the superposition of the transfer functions of all paths from receiver \(k\) to target location \(\mathbf{r}_j\) at frequency \(\omega_i\).


\section{Implementation of virtual targets for multipath wave propagation}\label{section:virtual_targets}

\section{Subdivision into wavefronts}

% !TeX root = ../main.tex
% Add the above to each chapter to make compiling the PDF easier in some editors.

\chapter{Results}\label{chapter:results}
The proposed imaging algorithm is tested in various scenarios to evaluate its performance.
To quantify the performance of the algorithm, the quality of the reconstructed images and the needed computational time is evaluated.


The scenarios are categorized into pure Line-of-Sight (LOS), pure Non-Line-of-Sight (NLOS) and multipath scenarios.
As real-world measurements are not feasible for this thesis, the measurement data is generated by the wave simulation program `Altair Feko' using the Method of Moments (MoM).
The imaging domain (red) is defined as the set of 256\(\times \)256 linearly distributed points in the x-y-plane.
Multiple active source-dipoles (blue) are placed in the imaging domain and their scattered field is recorded at 10000 receiver locations (green) in the y-z-plane for 20 logarithmically spaced frequencies between 4.8GHz and 7.2GHz.
This general setup is used for all scenarios as visualized in Figure~\ref{fig:general_setup}.

\begin{figure}[ht]
    \centering
    \includegraphics[width=0.8\textwidth]{figures/general_setup.pdf}
    \caption{The General Setup}\label{fig:general_setup}
\end{figure}


The imaging algorithm is then applied to the recorded data to reconstruct the sources in the imaging domain.


\section{Pure Line-of-Sight (LOS) Scenarios}
Pure Line-of-Sight scenarios are the simplest case for the imaging algorithm.
The setup for this scenario is exactly as shown in Figure~\ref{fig:general_setup}.
The source and the receiver are in direct line of sight, so there is only one wavefront that travels from the receivers to the imaging domain without any reflections.
If there are furthermore no refractions (homogenous medium with refractive index \(n\)), the algorithm boils down to the following formula:
\begin{equation}\label{eq:naive_algorithm}
    I(\bm{r}_k) = |\sum_{i=1}^{N} \sum_{j=1}^{M} \frac{\underline{\mathsf{E}}_{ij}^*}{|\bm{r}_j - \bm{r}_k|} \cdot \mathrm{e}^{\mathrm{i}k_0\cdot n \cdot |\bm{r}_j - \bm{r}_k|}|
\end{equation}

\begin{figure}[ht]
    \centering
    \includegraphics[width=0.8\textwidth]{figures/los_result.pdf}
    \caption{The image of the pure LOS-scenario}\label{fig:los_results}
\end{figure}

The calculated image for this scenario is shown in Figure~\ref{fig:los_results}.
The emitting dipole sources can be easily localized.
This image would also result from evaluating Equation~\eqref{eq:naive_algorithm} at every point in the imaging domain and coloring the imaging point accordingly.
To do this no knowledge about the medium or the setup is needed and only the distances between the receivers and the imaging points are considered.
This approach is called the naive algorithm and is used as a reference to compare the proposed algorithm to.


\section{Pure Non-Line-of-Sight (NLOS) Scenarios}
To create a pure Non-Line-of-Sight scenario, a PEC-plate is placed between the source and the receiver to block the direct path.
Another PEC-plate is placed perpendicular on top of the blocking plate with a small gap between both plates to create a reflecting path.
The resulting setup is shown in Figure~\ref{fig:nlos_setup}.

\begin{figure}[ht]
    \centering
    \includegraphics[width=0.8\textwidth]{figures/nlos_setup.pdf}
    \caption{The NLOS Setup}\label{fig:nlos_setup}
\end{figure}

In this setup only a small fraction of the emitted wave will pass through the gap, reflect off the top plate and reach the receivers.
A simpler algorithm that only considers the distances between the receivers and the imaging points would fail to reconstruct the dipole locations.
The proposed algorithm on the other hand is able to reconstruct the sources as shown in Figure~\ref{fig:nlos_results}.
In the plots the paths of the rays that are considered by the algorithm are visualized as yellow lines for some of the imaging points.

\begin{figure}[ht]
    \centering
    \includegraphics[width=0.49\textwidth]{figures/nlos_naive_result.pdf}
    \includegraphics[width=0.49\textwidth]{figures/nlos_result.pdf}
    \caption{The Intensity of the Backpropagated field using a naive algorithm (left) compared to the ray-tracing algorithm (right)}\label{fig:nlos_results}
\end{figure}

This scenario with only one reflection off a plane PEC-plate is very similar to the pure LOS-scenario.
There is no multipath propagation between the source and the receiver, so the algorithm can be simplified to a similar form as Equation~\eqref{eq:naive_algorithm} with the only difference being that the distance between the imaging point and the \textbf{reflected} receiver-location should be considered.



\section{Multipath Scenarios}
The initial idea for using the ray-tracing algorithm was to improve the imaging for multipath scenarios.
The resulting image should theoretically enhance if the geometry of the setup and therefore all the paths between the imaging domain and receivers are taken into account, as more information is available than in a naive approach.
To test this, the following two scenarios are considered (see Figure~\ref{fig:MultipathNLOS_setup}).
First a setup with multiple reflecting paths and no direct path between the source and the receiver is examined.
Secondly the same setup is used, but the blocking plate is removed so a direct path between the source and the receiver exists.

\begin{figure}[ht]
    \centering
    \includegraphics[width=0.8\textwidth]{figures/MultipathNLOS_setup.pdf}
    \caption{The Multipath Setup}\label{fig:MultipathNLOS_setup}
\end{figure}

\subsection{NLOS Multipath}
As described in Section~\ref{section:subdivision_into_wavefronts}, an image for every separate wavefront between the imaging domain and receivers is produced by the algorithm.
To show, that the algorithm actually improves by considering all the paths, the image for every wavefront is compared to their combined result which is retrieved by summing up the images of every wavefront.



\subsection{Mixed Multipath}

\section{Limitations}